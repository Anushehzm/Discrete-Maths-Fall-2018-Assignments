\documentclass{article}
\usepackage{graphicx,fancyhdr,amsmath,amssymb,amsthm,subfig,url,hyperref}
\usepackage[margin=1in]{geometry}
\newtheorem{theorem}{Theorem}


%----------------------- Student and Homework Information --------------------------

%%% PLEASE FILL THIS OUT WITH YOUR INFORMATION
\newcommand{\myname}{Anusheh Zohair Mustafeez}
\newcommand{\myid}{21100072}
\newcommand{\hwNo}{Homework 8}
%%% END



\fancypagestyle{plain}{}
\pagestyle{fancy}
\fancyhf{}
\fancyhead[RO,LE]{\sffamily\bfseries\large LUMS}
\fancyhead[LO,RE]{\sffamily\bfseries\large CS-210 Discrete Mathematics}
\fancyfoot[LO,RE]{\sffamily\bfseries\large \myname: \myid @lums.edu.pk}
\fancyfoot[RO,LE]{\sffamily\bfseries\thepage}
\renewcommand{\headrulewidth}{1pt}
\renewcommand{\footrulewidth}{1pt}

%--------------------- This is the title of the document. DO NOT CHANGE IT ------------------------

\title{CS-210 \hwNo}
\author{\myname \qquad Student ID: \myid}

%--------------------------------- AFTER Entering the Student and Homework Information, write your answers below  ----------------------------------

\begin{document}
\maketitle


\section{Problem 3}
Collaborated with: Hajira Zaman (21100057) and Safah Barak (21100092) \\
We know that G = (V,E) $\rightarrow$ $\bar{G}$=(V,$\bar{E}$)\\
So if we take two vertices x and y such that (x,y) $\in$ E then (x,y) $\notin$ $\bar{E}$ so this means that if an edge incident on x and y is part of G, it will not be incident on x and y in $\bar{G}$ and vice versa.\\ 
Case 1:\\
If (x,y) is a edge that is missing in the disconnected G, it will be part of $\bar{G}$ making $\bar{G}$ a connected graph such that now there will a path from every vertex to another.\\
Case 2:\\
If (x,y) forms an edge in G then both x and y are part of the same connected component in G. In this case, if we have another vertex z which is part of another connected component of G then (x,z) and (y,z) are edges missing in G as it is disconnected. So (x,z) and (y,z) will form edges in $\bar{G}$ so now eventhough x and y are not adjacent, they still have a path between them. In this way all vertices will have paths between them making $\bar{G}$ a connected graph in this case as well.\\

\section{Problem 6}
Collaborated with: Hajira Zaman (21100057) and Safah Barak (21100092)\\
If F is a forest with k components and each component has $v_{1},v_{2},v_{3}....v_{k}$ vertices respectively then each component will have  $v_{1}-1,v_{2}-1,v_{3}-1....v_{k}-1$ edges respectively too. If we take n to be the order of F and e to be the size of F, then:\\ n = $\sum_{m=1}^{k} v_{m}$ and e = $\sum_{m=1}^{k} v_{m}-1$ \\ So e = $\sum_{m=1}^{k} v_{m}$ - $\sum_{m=1}^{k} 1$. \\ And finally e= n - k (proved). 

\section{Problem 10}
Collaborated with: Hajira Zaman (21100057) and Safah Barak (21100092)\\
\begin{enumerate}
\item %A
If we take the longest path P in a graph G such that the vertices in that path include $v_{1},v_{2},v_{3}....v_{n}$ then the length of P is n -1 where n is the size of the path. Taking vertex $v_{1}$ into account we know that the maximum number of degree of $v_{1}$ is n - 1  considering it to be connected to all other vertices in path P. 
Now assuming n - 1 $<$ k but the question tells us that each degree should be $\geq$ k and of we consider the degree of $v_{1}$ this means n -1 $\geq$ k so we have a contradiction hence proved. 

\item %B
Since all our vertices are connected to vertices within P, the smallest cycle we can have with $v_{1}$ is v1-v2-v3-v1 in which case $v_{1}$ will have degree 2 and length of path will be 3. If we consider the second biggest cycle from v1 to v4 also then the length of path from $v_{1}$ will go up by 1. I we continue doing this upto the nth and final vertex, the length of the path will be k +1 proving that there will be at least one path of length k + 1. 

\end{enumerate}

\section{Problem 12}
Collaborated with: Hajira Zaman (21100057) and Safah Barak (21100092)\\
If T is a full m-ary tree with height 3 and m is a positive integer, then it will have a total of four levels such that the number of nodes at each level are $m^{level-1}$. We are only interested in the leaves i.e. are the nodes at the lowest level which is level four, we will see that the number of leaves will be $m^{4-1}$ = $m^{3}$. If $m^{3}$ = 84 then m cannot be an integer hence we have a contradiction proving that there cannot be a full m-ary tree with height 3 and 84 leaves. 
 
\section{Problem 14}
Collaborated with: None\\
The handshaking lemma tells us that the sum of the degrees of vertices in a graph is even and that the number of odd degree vertices is also even. \\
Taking this into account if a simple graph has an odd vertex x, then it cannot exist alone as 1 is not an even number and the graph needs to have an even number of odd degree vertices. \\ 
Another way of looking at it is that any path that contains x is a graph in itself and this graph needs to have an even sum of degrees. To achieve this we need at least one vertex of odd degree in that path as the sum of even and odd is odd while the sum of odd and odd is even. Hence shown that there exists a path from any vertex of odd degree to some other vertex of odd degree in a simple graph.


\section{Problem 15}
Collaborated with: Hajira Zaman (21100057) and Safah Barak (21100092)\\
The maximum number of edges can exist only when all components are complete graphs so if a graph has $v_{m}$ vertices then it will have ($(v_{m})(v_{m}-1)$)/2 edges. \\Using a proof similar to that in Problem 6, we can see that the maximum number of edges is  $\sum_{m=1}^{k} v_{m}-1$ = n - k\\
Squaring both sides we get:
 ($\sum_{j=1}^{k} v_{j}-1)(\sum_{m=1}^{k} v_{m}-1)$ = $n^{2} - 2nk + k^{2}$ \\
 $\sum_{m=1}^{k} ((v_{m})^{2}- 2v_{m} + k)$ + a nonnegative term due to the double summation = $n^{2} - 2nk + k^{2}$  \\
So  $\sum_{m=1}^{k} ((v_{m})^{2})$ - 2n + k $\leq$ $n^{2} - 2nk + k^{2}$  \\
$\sum_{m=1}^{k} (v_{m}^{2})$ $\leq$ 2n - k + $n^{2} - 2nk + k^{2}$ = $n^{2} - (k-1)(2n-k)$  \\
Now since the maximum size of all the components combined is $\sum_{m=1}^{k} ((v_{m})(v_{m}-1))/2 $ \\
Which is ($\sum_{m=1}^{k} (v_{m}^{2}$) - n)/2 \\
Plugging in the inequality e $\leq$ ($n^{2} - (k-1)(2n-k)$ -n)/2 \\
Hence proved e $\leq$ ((n-k)(n - k + 1))/2 

\section{Problem 19}
Collaborated with: None\\
A bipartite graph consists of even length cycles only and according to this the least number of edges needed for each face in a bipartite graph is 4. The Face-edge handshaking lemma tells us that: \\
2e= $\sum_{G \in R} deg(G)$ where R are the regions in graph G\\
Since the minumum degree in each region of a bipartite graph is 4, we get the inequality 2e $\geq$ 4f\\ 
The Euler's formula tells us: f - e + v = 2\\
Putting 2e $\geq$ 4f in the Euler's formula: \\
2/4 e - e + v $\geq$ 2 \\
-1/2 e +v $\geq$ 2 \\
2v - e $\geq$ 4\\
e $\leq$ 2v - 4

\end{document}
