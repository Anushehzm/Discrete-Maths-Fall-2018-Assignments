\documentclass{article}
\usepackage{graphicx,fancyhdr,amsmath,amssymb,amsthm,subfig,url,hyperref}
\usepackage[margin=1in]{geometry}
\newtheorem{theorem}{Theorem}


%----------------------- Student and Homework Information --------------------------

%%% PLEASE FILL THIS OUT WITH YOUR INFORMATION
\newcommand{\myname}{Anusheh Zohair Mustafeez}
\newcommand{\myid}{21100072}
\newcommand{\hwNo}{Homework 9}
%%% END



\fancypagestyle{plain}{}
\pagestyle{fancy}
\fancyhf{}
\fancyhead[RO,LE]{\sffamily\bfseries\large LUMS}
\fancyhead[LO,RE]{\sffamily\bfseries\large CS-210 Discrete Mathematics}
\fancyfoot[LO,RE]{\sffamily\bfseries\large \myname: \myid @lums.edu.pk}
\fancyfoot[RO,LE]{\sffamily\bfseries\thepage}
\renewcommand{\headrulewidth}{1pt}
\renewcommand{\footrulewidth}{1pt}

%--------------------- This is the title of the document. DO NOT CHANGE IT ------------------------

\title{CS-210 \hwNo}
\author{\myname \qquad Student ID: \myid}

%--------------------------------- AFTER Entering the Student and Homework Information, write your answers below  ----------------------------------

\begin{document}
\maketitle

\section{Problem 1}
Collaborated with: None\\
Proving: a mod m = b mod m iff m$|$(a-b) \\
m$|$(a-b) means $\exists$n $\in$ Z such that a-b = nm\\
The Division Theorem tells us: \\
Let c be an integer and d a positive integer, then there are unique integers q and r, with 0 $\leq$ r $<$ d such that c = dq + r\\
Since a and b are both integers and m a positive integer the Division Theorem will apply on them such that: \\
a=m$q_{1} + r_{1}$ and b=m$q_{2} + r_{2}$ where  0 $\leq$ $r_{1}$ $<$ m and  0 $\leq$ $r_{2}$ $<$ m\\
Putting this into a - b= nm: \\
a = b + nm \\
a = (m$q_{2} + r_{2}$) + nm\\
a = m ($q_{2}$ + n) + $r_{2}$\\
But we also know that a=m$q_{1} + r_{1}$ \\
So we can deduce that $r_{1}$ = $r_{2}$ and $q_{1}$ = $q_{2}$ + n \\
Since a and b have the same remainders when divided with m, they'll belong to the same residual class mod m and a mod m = b mod m\\
As the remainders are same we can write a and b in the forms a=m$q_{1} + r_{1}$ and b=m$q_{2} + r_{1}$ \\
So a - b = (m$q_{1} + r_{1}$) - (m$q_{2} + r_{1}$)\\
a-b= m ($q_{1} - q_{2}$)\\
a-b= m ($q_{2} + n - q_{2}$)\\
a-b= mn\\
Hence proved a mod m = b mod m iff m$|$(a-b)

\section{Problem 2}
Collaborated with: Safah Barak (21100092) and Hajira Zaman (21100057)
\begin{enumerate}
\item %A
If we take any integer x then the sum of three consecutive numbers with x as the first term will be: \\
x + (x+1) + (x+2) = 3x + 3 \\ = 3(x+1)\\
Since $\forall$x 3$|$3(x+1) as 3(x+1) = 3(y) where $\exists$y $\in$ Z such that y = x+1 hence the statement is proved.
\item %B
For any two even numbers a and b, a can be written as 2x and b as 2y\\
So the product of a and b can be written as 4xy \\
Since $\forall$x,y 4$|$4xy as 4xy = 4(z) where $\exists$z $\in$ Z such that z = xy hence the statement is proved.
\item %C
For any 4 consecutive integers any two alternate integers will be odd and the other two alternate will be even. If we take x to be the first even integer then x+2 will be the second. Since both are even we can write then as x=2a and x+2 = 2a + 2= 2(a+1). We can see that a and a+1 will have opposite parities. So one even integer from our four consecutive integers will be a product of 2 and an odd integer and the other will be a product of 2 with an even integer. From this the first will be divisible by 2 as we know: \\
$\forall$c 2$|$2c as 2c = 2(y) where $\exists$y $\in$ Z such that y = c\\
And for the second, 2 multiplied with any even integer will be divisible by 4 using the proof presented in part 2(b) \\
So $\forall$y 4$|$4d as 4d = 4(z) where $\exists$z $\in$ Z such that z = d \\
This means that we can write one even integer as 2s and the other as 4t\\
So the product of any 4 consecutive integers will be:\\
e*(e+1)*(e+2)*(e+3)=2s*(e+1)*4t*(e+3) if the first integer is even or e*2s*(e+2)*4t if the first integer is odd\\
In either case we will get a result in the form 8w where w is an integer which will always be divisible by 8 as \\
$\forall$w 8$|$8w as 8w = 8(z) where $\exists$z $\in$ Z such that z = w
\item %D
If we take a=5, b=3 and m=2 then (5-3) mod 2 =0 \\
But 5 mod 2 = 1 and 3 mod 2 = 1 which means they don't have remainders zero like a-b hence the statement is disproved. 
\item %E
%If n is an odd integer, then 3n + 3 is divisible by 6.
If n an is odd integer we can write it as 2k+1 where k is an integer \\
So 3n + 3 = 3(2k + 1) + 3 \\= 6k + 3 + 3 \\= 6k+6 \\= 6(k+1) \\
Since  $\forall$k 6$|$6(k+1) as 6(k+1) = 6(z) where $\exists$z $\in$ Z such that z = k+1 hence the statement is proved.
\end{enumerate}

\end{document}
